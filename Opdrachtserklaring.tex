\documentclass[a4paper,kul]{kulakarticle} 

\usepackage[utf8]{inputenc}
\usepackage[dutch]{babel}

\address{
	Project ontwerpen in ... \\
	promotor\\
	assistenten}
\date{Academiejaar 2020-2021}
\title{Opdrachtsverklaring}
\author{Bas C, Sofie m, Nele Eeckman}

\begin{document}
	\maketitle
	
	\section*{Inhoud project}
	- een bijgestelde formulering van de ontwerpopdracht, met numerieke ontwerpspecificaties waar van toepassing
	
	- een tijdsplanning, bv. aan de hand van een Gantt-chart
	
	To help you further with your plan, ask yourselves the following question:
	*Do you already have the required knowledge to complete the tasks you mentioned yourselves?
	*If no, how would you acquire this knowledge? (literature?)
	*How to integrate the algorithms in the framework?
	*What about testing the integration?
	*What about intermediate and final tests of the complete framework?
	*What are the specifics of the GUI/
	*Is it possible to gradually build up the GUI? What are the essential components?
	*Think about the complete framework. Say that you have the complete setup (mobile EEG + a laptop + ...) and you press 'Start', what happens? What chain of actions occurs? 
	
	Try to think about the whole framework and each individual component. Note that the tasks you have already outlined are already a good start.
	
	
	\section*{Planning}
	
\end{document}	
	